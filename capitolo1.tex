\section{Cosa sono i Bitcoin?}

Bitcoin è una valuta elettronica creata nel 2008 da Satoshi Nakamoto, uno pseudonimo dietro al quale non si sa ancora con la precisione chi si nasconde.
Con il termine Bitcoin viene denotata sia la rete che consente il possesso e il trasferimento delle monete, che le monete stesse. Per convenzione, Bitcoin si riferisce alla tecnologia della rete, mentre \textit{bitcoin} alla valuta stessa.\\
Come ogni valuta, i bitcoin possono essere trasferiti tramite gli utenti, grazie ad un protocollo bitcoin che viene utilizzato tramite la rete Internet. \\ Il protocollo bitcoin può essere eseguito su differenti dispositivi, in modo tale da permettere la fruibilità del servizio anche attraverso gli smartphones.\\
I bitcoins possono essere comprati, venduti e scambiati con altre valute, tramite degli organismi specializzati nel cambio di monete virtuali. In un certo senso, Bitcoin è la forma perfetta di denaro per Internet, dal momento che è estremamente veloce, sicuro e senza limiti.
A differenza delle altre valute, i bitcoin sono esclusivamente virtuali, dietro di essi non esistono monete fisiche.\\ Per trasferire tali bitcoin, ogni utente possiede un \textbf{wallet}, ovvero un \textit{portafoglio digitale}, il quale serve per memorizzare il denaro e fungere da riferimento per eventuali transazioni.

\section{Cos'è la Blockchain?}
\subsection{Caratteristiche}
\section{Le Transazioni}
\section{Grafo delle transazioni}
\section{Caratteristiche del grafo}