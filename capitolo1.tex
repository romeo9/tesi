% !TeX spellcheck = it_IT
\section{Cosa sono i Bitcoin?}

Bitcoin è una valuta elettronica creata nel 2008 da Satoshi Nakamoto, uno pseudonimo dietro al quale non si sa ancora con la precisione chi si nasconde.
Con il termine Bitcoin viene denotata sia la rete che consente il possesso e il trasferimento di denaro, sia la moneta. Per convenzione, Bitcoin si riferisce alla tecnologia della rete, mentre \textit{bitcoin} alla valuta stessa.

Come ogni valuta, i bitcoin possono essere trasferiti tramite gli utenti, grazie ad un protocollo che viene rispettato all'interno della rete Internet, il quale può essere eseguito su differenti dispositivi, in modo tale da permettere la fruibilità del servizio anche attraverso gli smartphones.
 
I bitcoins possono essere comprati, venduti e scambiati con altre valute, tramite degli organismi specializzati nel cambio di monete virtuali. In un certo senso, Bitcoin è la forma perfetta di denaro per Internet, dal momento che è estremamente veloce, sicuro e senza limiti.

A differenza delle altre valute, i bitcoin sono esclusivamente virtuali, dietro di essi non esistono monete fisiche. Tali bitcoin vengono coinvolti in transazioni da mittente a ricevente, i quali possiedono delle chiavi crittografiche pubbliche e private che servono per trasmettere e sbloccare la spesa dei bitcoin ricevuti.
Infatti, senza la chiave privata, chi riceve i bitcoin non può spenderli in nessun modo. Tali chiavi vengono conservate all'interno di un \textit{wallet}, letteralmente un "portafoglio". Ogni wallet è caratterizzato da un indirizzo Bitcoin il quale è univoco e ha la funzione di fare riferimento ad uno dei partecipanti alla transazione. In questo modo, quando viene effettuato uno scambio di bitcoin, vengono visualizzati solamente gli indirizzi dei wallet. Questa caratteristica permette quindi di rendere anonime le transazioni, dato che agli indirizzi non è connesso in nessun modo il nome o il cognome dell'individuo o dell'associazione che interviene nello scambio.

La rete Bitcoin, oltre ad essere completamente virtuale, è priva di un'unità centralizzata, infatti essa è costituita da un sistema distribuito peer-to-peer.\\
I bitcoin vengono creati tramite un processo, detto \textit{mining}, che permette a chiunque di mettersi in competizione per trovare una soluzione ad un problema matematico. Ogni persona che partecipa alla rete bitcoin, potrebbe operare come un \textit{miner}, ovvero colui che cerca di risolvere il problema matematico per generare bitcoin, usando le capacità del proprio computer messo a disposizione della computazione.\\

Quando viene effettuato uno scambio di bitcoin tra due o più wallet, viene creata una \textit{transazione}. Ogni transazione viene conservata in una struttura dati chiamata \textbf{blocco}, il quale a sua volta va a costituire la \textbf{Blockchain}.

\section{Cos'è la Blockchain?}
La Blockchain, letteralmente \textit{catena di blocchi}, è una base di dati distribuita, che permette la memorizzazione delle transazioni raggruppate in blocchi connessi tra loro, ognuno con il suo successivo. Ogni blocco è una struttura dati che contiene un numero variabile di transazioni, inserite dal \textit{miner} che ha minato il blocco, fino ad un tetto massimo di 1MB per singolo blocco.\\

L'idea di base della Blockchain, deriva dal concetto di \textbf{libro mastro}, il registro della contabilità in cui sono riuniti tutti i conti che compongono un dato sistema contabile. In questo caso il sistema contabile sarebbe la rete Bitcoin, mentre la Blockchain sarebbe il libro mastro che contiene tutti i conti, ovvero le transazioni. \\

L'antico libro mastro, veniva utilizzato come fonte ufficiale per la memorizzazione degli scambi e dei passaggi di proprietà. Infatti, quando veniva fatta una compravendita tra mittente e ricevente, veniva controllato sul libro mastro se il ricevente non avesse speso precedentemente il denaro, e se il mittente non avesse già venduto la merce usata nello scambio.

Infine, se fosse andato tutto a buon fine, veniva registrata la transazione sul libro mastro, in modo da essere consultabile e pubblica per le successive transazioni.

Un principio importante di questo meccanismo è la fiducia che tutti ripongono nel libro mastro: ognuno si fida del gestore della memorizzazione delle transazioni, al punto che, chi compra e chi vende, può effettuare scambi anche senza fidarsi reciprocamente. Quindi, il libro mastro è una garanzia, sia per il mittente che per il ricevente dello scambio. Inoltre, le banche possono perciò controllare gli scambi che vengono fatti e il denaro posseduto da ogni partecipante alle transazioni.



La Blockchain, come già sottolineato, è una struttura dati composta da diverse unità di base, dette blocchi. Infatti, blockchain significa letteralmente \textit{catena di blocchi}.

Quindi, tali blocchi vengono "incatenati", ovvero collegati tra loro tramite un protocollo ben definito nella struttura del sistema bitcoin. 

Ogni blocco all'interno della blockchain, è identificato da un codice hash, generato applicando l'algoritmo di crittografia SHA256 all'header del blocco. Un singolo blocco è collegato al suo predecessore, conosciuto come "blocco genitore", attraverso il campo \textit{previous block hash} all'interno del proprio header. In altre parole, ogni blocco contiene l'hash del proprio blocco genitore all'interno dell'header. Infine, la sequenza dei vari hash genera una catena che collega tutti i blocchi all'indietro, fino al blocco numero zero.

\subsection{Mining}

\subsection{Blockchain Forks}

Sebbene un blocco abbia solo un genitore, esso può avere temporaneamente un diverso numero di figli. Ogni figlio si riferisce allo stesso blocco genitore e contiene lo stesso hash del genitore nel campo  \textit{previous block hash}. Questa temporanea molteplicità di figli può causare una \textit{fork} (letteralmente \textit{biforcazione}), ovvero una situazione in cui blocchi differenti vengono creati quasi simultaneamente da diversi miners.

\includegraphics[width=0.8\linewidth]{figure/fork1}\\

\includegraphics[width=0.8\linewidth]{figure/fork2}\\

\includegraphics[width=0.8\linewidth]{figure/fork3}\\

\includegraphics[width=0.8\linewidth]{figure/fork4}\\

\includegraphics[width=0.8\linewidth]{figure/fork5}\\


%	Minare bitcoin
%		come si fa
%		chi lo fa
%		guadagno
%		complessità computazionale
%		merkle hash tree
	

\section{Le Transazioni}

Come è stato detto in precedenza, una transazione è uno scambio di monete bitcoin tra due o più individui. Per esempio, se Alice vuole dare 1BTC a Bob, ha bisogno di effettuare una transazione inserendo come input l'importo da trasferire e l'indirizzo del wallet di Bob. \\ 

Nella maggior parte dei casi, la quantità di denaro che partecipa alla transazione non è quasi mai
Secondo la configurazione predefinita dal protocollo, quando viene effettuata una transazione che non implica la spesa dell'intero importo contenuto nel wallet del mittente, è previsto che venga creata un ulteriore transazione con il 

Una caratteristica delle transazioni è che possono partecipare più wallet contemporaneamente: ad esempio, un individuo può inviare con la stessa transazione, una quantità di bitcoin a due individui diversi, in modo da recapitare ciascuno la quantità di bitcoin desiderata. 

\section{Grafo delle transazioni}
\section{Caratteristiche del grafo}