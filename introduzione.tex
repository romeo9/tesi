Negli ultimi anni, con il progresso tecnologico e con l'aumentare del coinvolgimento di Internet nella nostra quotidianità, si è arrivati a digitalizzare anche il denaro. Grazie a questa digitalizzazione, sono state create delle monete virtuali, o valute elettroniche, che possono essere utilizzate solo su Internet. La moneta elettronica che ha lanciato questa "tendenza" è una delle più famose, il Bitcoin.

Il Bitcoin non è solamente una valuta virtuale ma identifica tutta la rete che serve per la generazione e l'interscambio di essa. Infatti, la rete Bitcoin appartiene ad una di quelle nuove tecnologie che ancora non si conoscono a pieno. 

Una delle parti importanti della struttura della rete Bitcoin, è la Blockchain, un libro mastro virtuale che viene utilizzato per tenere traccia di tutte le transazioni tra un utente e un altro. Esso, ha la struttura di un database distribuito, dove ogni singolo nodo della rete, possiede una sua copia aggiornata. 

La Blockchain permette la memorizzazione delle transazioni attraverso delle strutture di dati, chiamate blocchi. Ogni blocco può contenere un numero arbitrario di transazioni. 

La particolarità di questi bitcoin è che per essere ottenuti, vanno "minati". Come si faceva in antichità con l'oro, che si estraeva dalle miniere (e quindi veniva "minato"), così viene fatto per i bitcoin. L'attività del minare, in questo caso però, viene effettuata risolvendo una gara crittografica attraverso la potenza computazionale della propria macchina. Chi riesce a risolvere tale gara, aggiungerà un nuovo blocco alla blockchain e otterrà un premio in bitcoin. Oltre al premio tuttavia, ogni transazione inserita all'interno di tale blocco pagherà delle tasse ("fee") a colui che ha minato il blocco.

Dal punto di vista economico, la rete Bitcoin non è vincolata a nessun tipo ti controllo bancario. Si presenta come una rete di interscambio di denaro, costituita da nodi peer-to-peer che cooperano tra di loro. Infatti, all'interno di ogni nodo viene salvata una copia della blockchain, ed al momento dell'inserimento di un nuovo blocco, viene aggiornata in tutta la rete.

Una delle caratteristiche che ha reso i bitcoin così celebri, è la possibilità di mantenere l'anonimato quando viene effettuata una transazione. Infatti le transazioni vengono effettuate tra due o più identificativi hash, che però non hanno un ufficiale riscontro sulla persona fisica che sta effettuando lo scambio di denaro.

A causa di queste caratteristiche, e grazie alla garanzia dell'anonimato, vengono fuori delle forme di attività illecita. Tali attività sono rese possibili grazie all'utilizzo dei bitcoin all'interno del deep web, con l'interfaccia Tor.

Esistono dei software, chiamati mixer, che svolgono la funzione di rendere più anonime le transazioni di un determinato utente. Per fare ciò, tali mixer, creano delle transazioni fittizie con lo scopo di miscelare i soldi e restituirli al mittente. Queste transazioni inoltre, vengono generate in modo sequenziale al punto da generare delle "chain" (catene) di transazioni.

Lo scopo di questo progetto è stato identificare le catene di transazioni automatiche, ovvero generate da mixer, piuttosto che quelle manuali, create da normali transazioni per servizi o prodotti.

Questo argomento viene affrontato nel paper \cite{ddp-ltcbh-17}, nel quale è stato definito un parametro chiamato "llc". Esso, assegnato ad ogni transazione, sta a rappresentare la lunghezza della catena di transazioni più lunghe a cui appartiene (llc = Lenght of Longest Chain). Utilizzando tale parametro per creare un grafico con la sua distribuzione sulle transazioni, si è riscontrato che i comportamenti automatici venivano rilevati, ma che c'era stata l'introduzione di alcuni \textit{bias} che non potevano essere corretti.
In questa tesi, si può osservare un altro approccio di analisi per gli stessi comportamenti, creando ulteriori modelli di studio per nuove variabili formulate. Infatti, invece del parametro llc, vengono individuate le nozioni di frequenza e di varianza, che probabilmente riescono meglio a soddisfare e a rendere l'output più corrispondente al risultato che si vuole ottenere. 

Nel primo capitolo viene approfondita la tecnologia bitcoin, la blockchain, i dettagli delle transazioni e come viene svolto il mining. 

Nel capitolo successivo, il capitolo 2, si parlerà del dettaglio dei comportamenti automatici, ovvero cosa sono i mixer e come generano le chain di transazioni.

Successivamente, in un altro capitolo, verranno affrontate le tematiche riguardanti il grafo delle transazioni, un grafo generato dalla concatenazione delle transazioni attraverso gli input e gli output.

Il capitolo 4 raccoglie tutta l'attività di questo periodo di tesi, dal dataset utilizzato, all'algoritmo che ha avuto un ruolo fondamentale nell'intero progetto.

Infine, il capitolo 5 cerca di mostrare i risultati ottenuti dall'esecuzione dell'algoritmo al variare del dataset, mettendoli a confronto con altri grafici ottenuti da altri studi nello stesso ambito.

