Le transazioni Bitcoin e il grafo che ne deriva, è stato e sarà largamente oggetto di studio. Le sue caratteristiche permettono infatti di studiarlo e promuovere nuove teorie su di esso.

In questo progetto ci si è limitati a studiare la frequenza e la varianza che caratterizzano le catene di transazioni più lunghe, ovvero quelle che molto probabilmente potrebbero essere state generate automaticamente da un software per il bitcoin mixing. 

Dopo aver visualizzato i dati ottenuti, l'andamento decrescente dei grafici nel capitolo precedente, potrebbe significare che esiste un numero maggiore di transazioni con frequenza molto piccola. Successivamente i grafici indicano che con l'aumentare del valore della frequenza, il numero di transazioni con quel valore rimane costante.

L'approccio utilizzato in questa tesi, si può considerare come uno sviluppo e un nuovo modo di interpretare le teorie promosse nel paper \cite{ddp-ltcbh-17}. Dai risultati si nota che, mentre nel paper veniva considerato un singolo parametro llc, qui invece intervengono numerosi fattori, tra cui la frequenza e la varianza.

Dal punto di vista finale, si nota che il secondo approccio potrebbe essere più performante del primo e potrebbe dar vita ad ulteriori studi su di esso. Ovviamente, non dev'essere visto come un'alternativa al paper, ma semplicemente un modo contemporaneamente affine e diverso di trovare la soluzione ad uno stesso problema. 

Andando avanti con lo studio di tale argomento, si potrebbe individuare una periodicità nei dati delle frequenze, che andrebbe ad indicare uno schema di produzione di tali catene di transazioni. Ciò, potrebbe essere preso come spunto per sviluppi futuri. 